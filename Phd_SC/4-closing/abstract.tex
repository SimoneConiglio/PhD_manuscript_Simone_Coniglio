\begin{vcenterpage}

\noindent\rule[2pt]{\textwidth}{0.5pt}

{\large\textbf{Résumé ---}}
    Dans cette thèse, un cadre d'optimisation topologique est développé pour améliorer la conception du mât réacteur, des supports du moteur et des nacelles. La conception optimale est obtenue en tenant en compte d’une contrainte de stress de von Mises et d’une exigence propre à la conception du moteur (c’est-à-dire une réduction de la variation des jeux en but d'aube du moteur en présence de charges de manœuvre de l’avion). Ce travail est divisé en trois parties principales. Dans la première partie, les techniques éléments finis sont passées en revue et mises en œuvre pour traiter le formalisme des superéléments et les modèles de grands espaces de conception. De plus, de nouvelles méthodes sont proposées pour traiter la liaison entre maillages à interfaces incohérentes. Dans la seconde partie, l'optimisation topologique basée sur l'approche SIMP dans un cadre eulérien est envisagée. Une stratégie multigrille est développée pour réduire le nombre d'opérations associé à la construction de la matrice de filtrage. Les défis associés aux formulations basées sur les stress sont également étudiés. Dans la troisième partie, les approches lagrangiennes d’optimisation topologique sont analysées pour les formulations à base de compliance ou à base de stress. Une nouvelle approche appelée "Generalized Geometry Projection" est proposée en tant que méthode unifiée pour la mise en œuvre de plusieurs approches lagrangiennes telles que le "Moving Morphable Components", le "Geometry Projection" et le "Moving Node Approach". La méthodologie de conception proposée a été validée sur plusieurs exemples 2D académiques, puis testée sur un modèle de moteur 3D générique (1,4 million de DDL et jusqu'à 500000 variables de conception). Le cadre proposé fournit un nouvel outil pour l’exploration de conceptions innovants visant à améliorer l’intégration du moteur à l’aile.
    
    
    

{\large\textbf{Mots clés :}}
   Optimisation topologique, Éléments finis, Collage de maillage, Multi-maillage géométrique, Geometry Projection, Moving Morphable Components, Moving Nodes Approaches.
\\
\noindent\rule[2pt]{\textwidth}{0.5pt}
\end{vcenterpage}
\newpage
\begin{vcenterpage}
\noindent\rule[2pt]{\textwidth}{0.5pt}
%\begin{center}
%{\large\textbf{Title in english\\}}
%\end{center}
{\large\textbf{Abstract ---}}
      In this PhD thesis, a topology optimization framework is developed to support the design of pylon, engine mounts and nacelle. The optimal design is achieved considering a von Mises stress constraint and a requirement specific to engine design (i.e. reducing tip-clearance variation under aircraft maneuvers loads).  This work is divided in three main parts. In the first part finite element techniques are reviewed and implemented to deal with superelement formalism and large design space models. Moreover novel methods are proposed to deal with tying of inconsistent mesh interfaces. In the second part topology optimization based on the SIMP approach is considered. A multigrid strategy is developed to reduce the computational burden associated with filter matrix construction. Challenges associated with stress based formulations are also investigated. In the third part Lagrangian approaches to topology optimization are analyzed for both compliance and stress based formulations. A novel approach called Generalized Geometry Projection is proposed as a unified method for the implementation of several Lagrangian approaches such as Moving Morphable Components, Geometry Projection and Moving Node Approach. The proposed design methodology was validated on several academic 2D examples and then tested on a generic 3D engine model (1.4 Millions of DOFs and up to 500 thousands of design variables). The proposed framework supplies a novel tool for the exploration of disruptive designs for the improvement of the engine-to-wing integration.  
    
{\large\textbf{Keywords:}}
   Topology Optimization, Finite Element Analysis, Mesh tying, Geometric Multigrid, Geometry Projection, Moving Morphable Componets, Moving Nodes Approaches.
\\
\noindent\rule[2pt]{\textwidth}{0.5pt}
\begin{center}
 Institut Clément Ader, 3 Rue Caroline Aigle\\
 Toulouse, France
\end{center}
\end{vcenterpage}

%%% Local Variables: 
%%% mode: latex
%%% TeX-master: "../phdthesis"
%%% End:

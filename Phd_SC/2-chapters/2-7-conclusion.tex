\chapter{Conclusions and perspectives}
\addstarredchapter{Conclusion}
\markboth{Conclusion}{Conclusion}
\label{chap:4}
The goal of this work consists in developing a methodology for the quantification of the impact of engine integration design on tip-clearance variations (related to fuel consumption) and for the exploration of innovative designs. Topology optimization is identified as the most suitable technique to deal with such a problem without making too strong assumptions on the final design.
Therefore, a topology optimization framework compatible with industrial applications is proposed. The main contributions of this thesis can be divided in three main topics that reflect the structure of this thesis. A first chapter deals with numerical challenges concerning the use of finite element analysis for the evaluation of tip-clearance variations and von Mises failure criteria. The second chapter shows how to include such analyses inside a classical topology optimization (SIMP) Eulerian framework. Finally, the third chapter demonstrates the use of Lagrangian approaches in topology optimization for both compliance and stress-based formulations.
All the 7 numerical challenges that were identified in the introduction are finally addressed:
\begin{enumerate}
	\item \textbf{Dealing with industrial models}\\
	The proposed topology optimization framework is compatible with Abaqus or Nastran engine finite element models. The technique adopted for this purpose is superelement formulation  which reduces the simulation and optimization computational burden but keeping the same accuracy of a simulation including the whole engine model. Tip clearance post processing is also shown to be compatible with this technique.
	\item \textbf{Dealing with complex design space geometries}\\
	The proposed finite element framework uses 8 node brick elements with full integration. These can form structured meshes that can easily accommodate complex design zone geometries. 
	\item \textbf{Dealing with non-consistent meshes}\\
	The design zone mesh and the engine mesh are not constrained to be coincident at their interface. This result is achieved through the implementation of RL-RBF, Internodes and Mortar techniques in the proposed framework. In the author's contribution \cite{coniglio2018weighted} two new strategies were proposed: the so-called Weighted Average Continuity Approach and moment correction. The first achieves a tradeoff between computational burden, algorithmic complexity and accuracy while the second enforces the balance of mechanical momentum of all reviewed strategies even for curved interfaces.
	\item \textbf{Efficiency and scalability}\\
	The use of iterative approaches with different multigrid preconditioners and smoothers is implemented. Several smoothers are compared for the engine use case. An investigation on tuning parameters is conducted to improve the solver efficiency. To avoid the selection of a new optimal value per iteration a Line-Search strategy is proposed and compared with the result of classic approaches.  All tested methods are compatible with distributed memory environments.
	\item \textbf{Employing Lagrangian Approaches}\\
	MMC, MNA and GP strategies are available in the literature.  In the author's contribution it was also shown that these formulations are particular cases of a unified formulation that was named Generalized Geometry Projection \cite{coniglio2019generalized}. The relationships between formulations' parameters and optimization related issues like the presence of saddle points in the optimization problem were also studied and practical solutions were provided. The proposed continuation strategy for stress-based topology with MNA approach can be used to achieve improved optima with small area of intermediate densities. 
	\item \textbf{Dealing with stress constraints}\\
	Stress constraints can be considered during the optimization thanks to the Unified Aggregation Relaxation approach. This helps achieving singular optima and reduces the computational burden associated with local stress constraints. Moreover, this strategy was combined with the MNA approach as a particular case of GGP, which also constitutes a novel contribution of this work.
	\item \textbf{Employing specific geometric primitives}\\
	The use of pylon like components that can change shape and number of panels/bars is rendered possible within the proposed GGP approach. In this way the solution is enforced to be compatible with standard manufacturing techniques. 
\end{enumerate}
The proposed methodology is tested on a demonstration engine model \cite{coniglio2019enginepylon}.
It was demonstrated that it is possible to include tip clearance variation criteria in the topology optimization loop, even in the very preliminary design phases. The only requirement for such analysis is the availability of a Whole Engine Model capable of post processing tip clearance variations.
Topology optimization using SIMP, MNA with rectangular parallelepiped components and MNA with pylon box components approaches provide different solutions. But all these solutions can be considered as variants of the same architecture (exhibiting the same load path). By making a parametric study on the value of the allowable fuel consumption variation it is also possible, using the SIMP approach, to identify a trend in the solution load path that can be used as design driver for integration engineers.
This work led to the publication of three journal papers \cite{coniglio2018weighted,coniglio2019enginepylon,coniglio2019generalized} and to two conference articles \cite{coniglio2017pylon,coniglio2018original}.
\section{Perspectives}
The proposed approach can be used to solve several engineering problems in which displacements control is required on a linear industrial model. The proposed library of geometric primitives can be enriched to treat different industrial problems. Still many developments could improve the proposed framework. 
The extension of the proposed framework to unstructured finite element mesh could give even more freedom to the shapes that can be considered for the design space.
Distributed Memory environments and GPU accelerators can be used to further refine the solution. This would also make it possible to consider larger design space and include several other subcomponents in the optimization, such as nacelle, air inlet and pylon secondary structures.
Optimization algorithms employed in this thesis were all local. In this context  it would be possible to consider improvement techniques such as tunneling \cite{zhang2018finding} aimed at avoiding that the algorithm gets stuck in local optima.
Surrogate models could also be employed to improve the overall optimization efficiency especially for very expensive models using Efficient Global Optimization (EGO) \cite{bouhlel2018efficient}. This could be effective especially for Lagrangian Approaches when the number of design variables can be reduced to less than 100.
Different formulations of stress relaxation such as the one considered in \cite{zhang2017stress} should be tested, to try to avoid convergence to nonsingular optima in general.
Other important physical analyses could be integrated in this framework such as dynamic analysis, non-linearity, thermo-mechanical analysis, aerodynamic or multi-physic analysis. For example, an application of such a framework including 2D nonlinear finite element model was investigated in a recent contribution for the design of a compliant mechanism of a morphing wing \cite{capasso2019optimisation}.
The main challenge that needs to be addressed to consider such simulations, is the fact that superelements, that require a linear static engine model to be employed, should be replaced by other techniques such as non-intrusive coupling \cite{Gendre2009}. Another major issue consists in the use of tying relationships at the interfaces. Engines being mounted under pylons and being subject to large temperature variations, need to be connected by a nearly isostatic system of engine mounts. Therefore, the solution proposed by the proposed framework could need some changes in the kinematic link that they have with the engine and this could deprecate their performances. Future works could address these issues to enforce more reliable results typically including the maximization of compliance induced by temperature variations load cases in the optimization formulation. 
Another interesting challenge for future work consists in making a topology optimization consistent with fatigue requirements in large structures. In fact, a common assumption of stress-based topology optimization is that allowable stress is known \textit{apriori} in each structural component and is not influenced by the solution shape or by the manufacturing process. Actually, in large structures, the number of parts used to make the final assembly is unknown \textit{apriori}, the technology adopted for the manufacturing process depends on the shape of the solution, the allowable are also local properties of the solution. On top of that this hypothesis also limits the application of technologies like 3D printing to structures where fatigue considerations are important. In fact, 3D printed structures can have large range of fatigue allowable stress that depends on the thermal history experienced by the material in each point, on the residual stresses and on the final surface roughness. These aspects depend on the manufacturing process that is a consequence of the design shape. Indeed, the link between design geometry and manufacturing process is not only essential to ensure that a design can be manufactured but also to efficiently design a 3D printed structure. This kind of correlation between material properties and manufacturing process is also valid for other technologies such as composite structures. An approach that ensures the existence of a manufacturing process thanks to geometric primitives that rely on 3D printing manufacturing process was proposed and implemented in \cite{vilas2019Une}. The main idea is to determine a design as the consequence of a manufacturing process. Therefore, optimization design variables can be directly those of the manufacturing process instead of geometry parameters. In the future such approach could be used to simulate manufacturing process to get real material properties such as fatigue allowable stress. Then these properties could be projected on the finite element model simulating the component operating conditions. In this way the design optimization could take in account the effect of manufacturing process on final allowable stress. This would then ensure more realistic fatigue life estimation for 3D printed designs.


%%% Local Variables: 
%%% mode: latex
%%% TeX-master: "../phdthesis"
%%% End:
